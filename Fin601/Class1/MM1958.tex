\documentclass[10pt,a4paper]{article}
\usepackage[utf8]{inputenc}
\usepackage{amsmath}
\usepackage{amsfonts}
\usepackage{amssymb}
\pagenumbering{arabic}
\author{David Zynda}
\title{MM, Trade-off Model, Stockholder and Bondholder Conflicts}


\begin{document}
\maketitle


\section{Modigliani and Miller 1958}

\textbf{What Question is the paper trying to answer?} \\

This paper analyzes, as the title implies, the cost of capital of a firm taking on a new project. In particular, it asserts that whether a firm funds a new project with debt or equity is irrelevant to the value of the firm. \\

Formerly, it was asserted that firms had an optimal amount of debt they could take on which would minimize the average cost of capital.f This is, in part, justified by formula $(16)$ on page 277. For any capitalization rate (earnings-price ratio on common stock) $i_k^* > r$ the value of the firm must tend to rise with debt. Then, it is shown by equation $(17)$ that cost of capital falls when increasing leverage to a ``relevant'' range. Beyond the relevant range, yield is believed to rise sharply as market discounts excessive trading on the equity. \\

Modigliani and Miller argue, instead, that there cannon be a commodity which consistently sells at more than one price in a market. This is a part of the proof of \textit{Proposition I} which states: \textit{the market value of any firm is independent of its capital structure and is given by capitalizing its expected return at the rate $\rho_k$ appropriate to its class.} \\

They argue that if such result did not hold, investors could buy and sell stocks and bonds to exchange one income stream for another stream. As these arbitrage opportunities are exploited, the markets will eventually clear. \\





\noindent \textbf{What is the economic theory or model?} \\

\noindent The value of a firm is described by: 

$$V_j = (S_j + D_j) = \bar{X}_j / \rho_k$$

for any firm $j$ in class $k$. \\

\noindent Then, the average cost of capital can be denoted as:

$$\frac{\bar{X}_j}{(S_j + D_j)} = \frac{\bar{X}_j}{V_j} = \rho_k$$


\noindent Now, Proposition II states that:

$$i_j = \rho_k + (\rho_k - r)D_j/S_j$$

where $i$ is the expected rate of return or yield on stock of company $j$. 

\noindent With taxes considered, we set total income $\bar{X}$ to:

$$\bar{X}_j^{\tau} = (\bar{X}_j - rD_j)(1 - \tau) + rD_j = \bar{\pi}_j^{\tau} + rD_j$$

Therefore, Prop 1 looks the same. But Prop 2 becomes:

$$i_j = \frac{\hat{\pi}_j^{\tau}}{S_j} = \rho_j^{\tau} + (\rho_k^{\tau} - r)D_j/S_j $$ 

\noindent Lastly, Prop III tells us that \textit{the cut-off point for investment in the firm will in all cases be $\rho_k$ and will be completely unaffected by the type of security used to finance the investment}. So, an investment is only worth taking if $\rho^*$, the rate of return on investment, is higher than $\rho_k$ (which equals the expected rate of return per share $\hat{x}_j$ divided by the price $p_j$). \\




\noindent \textbf{What does this paper contribute to the literature?} \\

Formerly, many corporate finance specialists were concerned with finding optimal levels of debt to maximize the value of their firm and lower the average cost of capital. This paper revolutionized this area by saying that was a largely worthless task. 





\section{Modigliani and Miller 1963 (A Correction)}

This paper adds a correction to the last. Originally, MM stated that the market values of firms in each class must be proportional in equilibrium to their expected returns net of taxes. However, it is the case contra this position, that the returns after taxes of two firms will not be proportional to their expected after-tax returns. Then, the tax benefits of taking on leverage will factor into the value of the firm, but still less so, MM argue, then the traditional view. The value of a levered firm of size $\hat{X}$ is then:

$$V_L = \frac{(1-\tau)\bar{X}}{\rho^{\tau}} + \frac{\tau R}{r} = V_U + \tau D_L$$

where $R$ is the interest bill. On page 439, we can see what happens when taxes are applied. 

\end{document}


\section{Miller 1977}